\documentclass{article}
\usepackage[utf8]{inputenc}
\usepackage{graphicx}
\usepackage{hyperref}
\hypersetup{colorlinks=true,urlcolor=blue,linkcolor=black, citecolor=black}
\usepackage[backend=biber, style=numeric]{biblatex} % other styles: numeric, authoryear, apa
\addbibresource{references.bib}
\usepackage{geometry}
\geometry{
    a4paper,
    left=20mm,
    right=20mm,
    top=20mm,
    bottom=20mm,
}

\begin{document}

\title{Individueel Projectplan}
\author{Seger Sars}
\date{\today}
\maketitle

\section{Projectdoel}
Het doel van dit project is om een 2D-renderer te maken die in staat is om verschillende dingen te renderen, zoals textures, tekst en vormen. 
Het doel is een renderer die lijkt op de SDL \cite{sdl} library, maar dan met een zelfbedachte API.
Ik ga deze renderer maken in C++ en daarna integreren in een game-engine die ik voor de systems programming minor heb gemaakt.
Als dit niet te lang duurt, ga ik daarna bezig met optimalisatie en/of het toevoegen van post-processing-effecten zoals bloom, motion blur of anti-aliasing.
Voor optimalisatie zijn er een aantal dingen die ik kan doen, zoals het gebruik van een texture atlas en batch rendering.

\section{Stappenplan}
% hieronder lijst van stappen niet per week maar per stap
Dit is een lijst van stappen die ik ga volgen tijdens dit project. 
Ik weet niet zeker of ik aan de post-processing-effecten ga toekomen, maar ik wil ze wel proberen toe te voegen.
\begin{itemize}
    \item Bekend raken met OpenGL met behulp van learnopengl.com \cite{learnopengl}. Omdat ik OpenGL wil leren in C++, lijkt learnopengl een goede keuze.
    \item API definiëren voor de basisfeatures van de renderer.
    \item Implementeren van de basisfeatures in de renderer.
    \item Integreren van de renderer in de game-engine.
    \item Performancetests uitvoeren om de prestaties te meten en te vergelijken met de SDL-renderer-implementatie.
    \item Optimalisatie van de renderer door gebruik te maken van een texture atlas en batch rendering.
    \item Toevoegen van enkele post-processing-effecten zoals bloom, motion blur of anti-aliasing.
\end{itemize}

\section{Bevindingen}
Hier komen belangrijke bevindingen te staan die ik tegenkom tijdens het project.


\section{Repository}
Mijn indproj-repository \cite{my_repo} is te vinden op GitHub.
De engine die ik heb gemaakt voor de systems programming minor is ook te vinden op GitHub \cite{engine_bravo}.
Daarnaast moet er nog een fork worden gemaakt van de engine waarin ik de renderer ga integreren.

\newpage

\section{Referenties}
\printbibliography

\end{document}
