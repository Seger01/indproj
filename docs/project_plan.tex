\documentclass{article}
\usepackage[utf8]{inputenc}
\usepackage{graphicx}
\usepackage{hyperref}
\hypersetup{colorlinks=true,urlcolor=blue,linkcolor=black, citecolor=black}
\usepackage[backend=biber, style=numeric]{biblatex} % other styles: numeric, authoryear, apa
\addbibresource{references.bib}
\usepackage{geometry}
\geometry{
    a4paper,
    left=20mm,
    right=20mm,
    top=20mm,
    bottom=20mm,
}

\begin{document}

\title{Individueel Project Plan}
\author{Seger Sars}
\date{\today}
\maketitle

\section{Project Doel}
Het doel van dit project is om een 2D renderer te maken die in staat is om verschillende dingen te renderer zoals textures, text, en shapes. 
Het doel is een renderer die lijkt op de SDL \cite{sdl} library maar dan met een eigen bedachte API.
Ik ga deze renderer maken in C++ en daarna integreren in een game engine die ik voor de systems programming minor heb gemaakt.
Als dit niet te lang duurt ga ik daarna bezig met optimalisatie en/of het toevoegen van post-processing effecten zoals bloom, motion blur, of anti-aliasing.
Voor optimalisatie zijn er een aantal dingen die ik kan doen zoals het gebruik van een texture atlas en batch rendering.

\section{Stappen plan}
% hieronder lijst van stappen niet per week maar per stap
Dit is een lijst van stappen die ik ga volgen tijdens dit project. 
Ik weet niet zeker of ik aan de post processing effecten ga komen, maar ik wil ze wel proberen toe te voegen.
\begin{itemize}
    \item Bekend worden met OpenGL met behulp van learopenopengl.com \cite{learnopengl}. Omdat ik opengl wil leren in C++, lijkt learnopengl een goede keuze.
    \item API definiëren voor de basis features van de renderer.
    \item Implementeren van de basis features in de renderer.
    \item Integreren van de renderer in de game engine.
    \item Performance tests uitvoeren om performance te meten en vergelijken met de SDL renderer implemntatie.
    \item Optimalisatie van de renderer door gebruik te maken van een texture atlas en batch rendering.
    \item Toevoegen van sommige post-processing effecten zoals bloom, motion blur, of anti-aliasing.
\end{itemize}

\newpage

\section{References}
\printbibliography

\end{document}
